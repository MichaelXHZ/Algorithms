\documentclass[12pt]{article}

\usepackage{algo,fullpage,url,amssymb,epsfig,color,xspace}
\usepackage[
pdftitle={CS 240 Assignment 1},
pdfsubject={University of Waterloo, CS 240, Fall 2016},
pdfauthor={Arne Storjohan}]
{hyperref}

\renewcommand{\thesubsection}{Problem \arabic{subsection}}

\begin{document}

\begin{center}
{\Large\bf University of Waterloo}\\
\vspace{3mm}
{\Large\bf CS240 - Fall 2016}\\
\vspace{2mm}
{\Large\bf Assignment 1 Part 1}\\
\vspace{3mm}
\textbf{Due Date: Wednesday September 21 at 5:00pm}
\end{center}

\definecolor{care}{rgb}{0,0,0}
\def\question#1{\item[\bf #1.]}
\def\part#1{\item[\bf #1)]}
\newcommand{\pc}[1]{\mbox{\textbf{#1}}} % pseudocode

Please read
\url{http://www.student.cs.uwaterloo.ca/~cs240/f16/guidelines.pdf}
for guidelines on submission.  Problems 1 -- 4 are written
problems; submit your solutions electronically as a PDF with file
name {\tt a01p1wp.pdf} using MarkUs. We will also accept individual
question files named {\tt a01q1w.pdf}, {\tt a01q2w.pdf}, ... , {\tt
a01q4w.pdf} if you wish to submit questions as you complete them.

Note: Assignment 1 has been split into 2 parts both worth 2.5\% each. 
Keep in mind that part 1 is due before part 2. 
There are 39 marks for part 1 and 34 marks for part 2. 
%%%%%%%%%%%%%%%%%%%%%%%%%%%%%%%%%%%%%%%%%%%%%%%%%%%%%%%%%%%%%
\subsection{[3+3+3+3+3=15 marks]}
Provide a complete proof of the following statements from first
principles (i.e., using the original definitions of order notation).
All logarithms are natural logarithms: $\log = \ln$.\\

\begin{minipage}[t]{24cm}
\begin{itemize}
\part{a} $12 n^3 +11n^2+10 \in O(n^3)$
\\\textbf{Solution: } 
\\We need to show that $\exists c$ and $\exists {n_0}$ such that $0 \leq 12 n^3 +11n^2+10 \leq c \cdot n^3$ for all $n \geq {n_0}$.
\\Obviously, $12 n^3 + 11 n^2 + 10 \leq 33 n ^3$ when $n \geq 1$
\\Thus, the given statement is true such that $c = 33$ and $n_0 = 1$
\part{b} $12 n^3 +11n^2+10 \in \Omega(n^3)$
\\\textbf{Solution: }
\\We need to show that $\exists c$ and $\exists {n_0}$ such that $0 \leq c\cdot n^3 \leq 12 n^3 + 11n^2 + 10$ for all $n \geq {n_0}$.
\\Obviously, $12 n^3 + 11 n^2 + 10 \geq 12 n ^3$ when $n \geq 1$
\\Thus, the given statement is true such that $c = 33$ and $n_0 = 1$
\part{c} $12 n^3 +11n^2+10 \in \Theta(n^3)$
\\\textbf{Solution: }
\\Since (a) and (b) have been proved to be true, $12 n^3 + 11n ^2 + 10 \in \Theta(n^3)$ is true by the fact.
\part{d} $1000n \in o(n \log n)$
\\\textbf{Solution: }
\\We need to show that $\forall c$ and $\exists {n_0}$ such that $0 \leq 1000n < c \cdot n \log n$ for all $n \geq {n_0}$.
\\Because $1000n < c \cdot n \log n \Rightarrow 1000 < c \cdot \log n \Rightarrow \frac{1000} {c} < \log n \Rightarrow e^\frac {1000} {n} < n$
\\Thus, $n_0 = e^\frac{1000} {n}$.
\\To prove that, plug $n_0$ in the equation:
\\$c \cdot n \log n > c \cdot n \log n_0 = c \cdot n \log e^\frac{1000} {c} = cn(1000/c) = 1000n$
\\So, the proof is done.
\part{e} $n^{n} \in \omega(n^{20})$
\\\textbf{Solution: }
\\We need to show that $\forall c$ and $\exists {n_0}$ such that $0 \leq c \cdot n^{20} < n^n$ for all $n \geq {n_0}$.
\\$cn^{20} < n^n \Rightarrow \log c + 20 \log n < n \log n = (n-1) \log n + \log n$
\\First, when $\log n > \log c \Rightarrow n > c$.
\\Second, when $(n-1) \log n > 20 \log n \Rightarrow n-1 > 20 \Rightarrow n > 21$.
\\Thus, $n^n > cn^{20}$ when $n > max(c,21)$.
\end{itemize}
\end{minipage}
%%%%%%%%%%%%%%%%%%%%%%%%%%%%%%%%%%%%%%%%%%%%%%%%%%%%%%%%%%%%%
\subsection{[4+4=8 marks]} 
For each pair of the following functions, fill in the correct asymptotic
notation among $\Theta$, $o$, and $\omega$ in the statement $f(n)\in
\sqcup(g(n))$.  Provide a brief justification of your answers.  In your
justification you may use any relationship or technique that is described
in class.
\begin{itemize}
\part{a} $f(n)=\sqrt{n}$ versus $g(n)=(\log{n})^{2}$
\\\textbf{Solution: }
\\
\\$lim_{n \rightarrow \infty} \frac{(\log n)^2} {\sqrt n}$
\\
\\Using L'Hopital's rule,
\\
\\$=lim_{n \rightarrow \infty} \frac{2 \cdot \log n \cdot \frac {1} {n}} {0.5n^{-0.5}}$
\\
\\$=lim_{n \rightarrow \infty} \frac{2 \cdot \log n} {0.5n^{0.5}}$
\\
\\$=lim_{n \rightarrow \infty} \frac{2 \cdot \frac {1} {n}} {0.25n^{-0.5}}$
\\
\\$=lim_{n \rightarrow \infty} \frac{2} {0.25n^{0.5}}$
\\
\\$= 0$
\\$\therefore f(n) \in o(g(n))$.
\part{b} $f(n)=n^3(5+2\cos{2n})$ versus $g(n)=3n^2+4n^3+5n$
\\\textbf{Solution: }
\\$\because -2 \leq 2\cos{2n} \leq 2$, $3 \leq 5+2\cos{2n} \leq 7$, $3n^3 \leq n^3(5+2\cos{2n}) \leq 7n^3$.
\\$\therefore f(n) \in \theta (n^3)$.
\\$\because \theta (3n^2+4n^3+5n) = \theta(4n^3) = \theta(n^3)$ by "Maximum Rules", $g(n) \in \theta (n^3)$.
\\$\therefore f(n) \in \theta (g(n))$.
\end{itemize}
%%%%%%%%%%%%%%%%%%%%%%%%%%%%%%%%%%%%%%%%%%%%%%%%%%%%%%%%%%%%%
\subsection{[6+6=12 marks]}
Prove or disprove each of the following statements.  To prove a
statement, you should provide a formal proof that is based on the
definitions of the order notations.  To disprove a statement, you can
either provide a counter example and explain it or provide a formal proof.
All functions are positive functions.
\begin{itemize}
\part{a} $f(n) \not \in o(g(n))$ and $f(n) \not \in \omega(g(n))
\Rightarrow f(n) \in \Theta(g(n))$
\\
\\\textbf{Solution:}
\\
\\$f(n) \not \in o(g(n))$
\\
\\$\Rightarrow$ not ($\forall c > 0, \exists n_0 > 0$ s.t. $0 \leq f(n) < cg(n)$ for all $n \geq n_0$
\\
\\$\Rightarrow \exists c > 0, \exists n > n_0, \forall n_0 > 0, f(n) < 0$ or $f(n) \geq cg(n)$
\\
\\$\because g(n) \geq 0$ and $f(n) \geq 0$
\\
\\$\therefore$ We choose $f(n) \geq cg(n) \geq 0$
\\
\\$\Rightarrow \exists c > 0, \exists n_0 > 0$ s.t. $f(n) \geq cg(n) \geq 0$,  $\forall n \geq n_0$
\\
\\$\Rightarrow f(n) \in \Omega(g(n))$
\\
\\$f(n) \not \in \omega(g(n))$
\\
\\$\Rightarrow$ not ($\forall c > 0, \exists n_0 > 0$ s.t. $0 \leq cg(n) < f(n)$ for all $n \geq n_0$
\\
\\$\Rightarrow \exists c > 0, \exists n > n_0, \forall n_0 > 0, cg(n) < 0$ or $cg(n) \geq f(n)$
\\
\\$\because g(n) \geq 0$ and $f(n) \geq 0$
\\
\\$\therefore$ We choose $cg(n) \geq f(n) \geq 0$
\\
\\$\Rightarrow \exists c > 0, \exists n_0 > 0$ s.t. $cg(n) \geq f(n) \geq 0$,  $\forall n \geq n_0$
\\
\\$\Rightarrow f(n) \in O(g(n))$
\\
\\Therefore, $f(n) \in \Theta(g(n))$.
\part{b} $\min(f(n),g(n)) \in \Theta\left (\frac{f(n)g(n)}{f(n)+g(n)}\right)$  
\\
\\\textbf{Solution: }
\\
\\Firstly, we need to prove $min(f(n),g(n)) \in O(\frac{f(n)g(n)} {f(n)+g(n)})$.
\\
\\We need to show that, $\exists c > 0$, $\exists n_0 > 0$, s.t. $0 \leq min(f(n), g(n)) \leq c \cdot \frac{f(n)g(n)} {f(n)+g(n)}$ for all $n \geq n_0$.
\\
\\If $f(n) > g(n)$, $\Rightarrow g(n) \leq c \cdot \frac{f(n)g(n)} {f(n)+g(n)} \Rightarrow 1 \leq \frac{cf(n)} {f(n) + g(n)} \Rightarrow$
\\
\\$ cf(n) \geq f(n) + g(n) \Rightarrow cf(n) \geq 2f(n) \geq f(n) + g(n) \Rightarrow$ $c \geq 2$ and $ n_0 \geq 1$
\\
\\If $f(n) < g(n)$, $\Rightarrow f(n) \leq c \cdot \frac{f(n)g(n)} {f(n)+g(n)} \Rightarrow 1 \leq \frac{cg(n)} {f(n) + g(n)} \Rightarrow$
\\
\\$ cg(n) \geq f(n) + g(n) \Rightarrow cg(n) \geq 2g(n) \geq f(n) + g(n) \Rightarrow$ $c \geq 2$ and $ n_0 \geq 1$
\\
\\Thus, the first part is proved.
\\
\\Secondly, we need to prove $min(f(n),g(n)) \in \Omega(\frac{f(n)g(n)} {f(n)+g(n)})$.
\\
\\We need to show that $\exists c > 0$, $\exists n_0 > 0$, s.t. $0 \leq c \cdot \frac{f(n)g(n)} {f(n)+g(n)} \leq min(f(n), g(n))$ for all $n \geq n_0$.
\\
\\If $f(n) > g(n)$, $\Rightarrow g(n) \geq c \cdot \frac{f(n)g(n)} {f(n)+g(n)} \Rightarrow 1 \geq \frac{cf(n)} {f(n) + g(n)} \Rightarrow$
\\
\\$ cf(n) \leq f(n) + g(n) \Rightarrow cf(n) \leq f(n) \leq f(n) + g(n) \Rightarrow$ $c \leq 1$ and $ n_0 \geq 1$
\\
\\If $f(n) < g(n)$, $\Rightarrow f(n) \geq c \cdot \frac{f(n)g(n)} {f(n)+g(n)} \Rightarrow 1 \geq \frac{cg(n)} {f(n) + g(n)} \Rightarrow$
\\
\\$ cg(n) \leq f(n) + g(n) \Rightarrow cg(n) \leq g(n) \leq f(n) + g(n) \Rightarrow$ $c \leq 1$ and $ n_0 \geq 1$
\\
\\Thus, the second part is proved.
\\
\\Therefore, $\min(f(n),g(n)) \in \Theta\left (\frac{f(n)g(n)}{f(n)+g(n)}\right)$ is true.
\end{itemize}
%%%%%%%%%%%%%%%%%%%%%%%%%%%%%%%%%%%%%%%%%%%%%%%%%%%%%%%%%%%%%  
\subsection{[4 marks]}
Derive a closed form for the following sum:
$$
S(n) = \sum_{i=1}^n i/2^i.
$$
\\\textbf{Solution: }
\\
\\$S(n) = 1/2 + 2/4 + 3/8 + 4/16 + ... + n/2^n$
\\
\\$= (1/2 + 1/4 + 1/8 + 1/16 + ...) + (1/4 + 1/8 + 1/16 + 1/32 + ...) + (1/8 + 1/16 + 1/32 + 1/64 + ...) + ... + (1/2^n)$
\\
\\$= \frac{1/2} {1/2} + \frac{1/4} {1/2} + \frac{1/8}{1/2} + ... + (1/2^n)$
\\
\\$= 1 + 1/2 + 1/4 + 1/8 + 1/16 + ... + 1/2^n$
\\
\\$=\frac{1}{1/2}
\\
\\$=2

%%%%%%%%%%%%%%%%%%%%%%%%%%%%%%%%%%%%%%%%%%%%%%%%%%%%%%%%%%%%%

\end{document}
