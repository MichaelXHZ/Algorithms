\documentclass[12pt]{article}

\usepackage{algo,fullpage,url,amssymb,epsfig,color,xspace}
\usepackage[
pdftitle={CS 240 Assignment 1},
pdfsubject={University of Waterloo, CS 240, Fall 2016},
pdfauthor={Arne Storjohan}]
{hyperref}

\renewcommand{\thesubsection}{Problem \arabic{subsection}}
\begin{document}

\begin{center}
{\Large\bf University of Waterloo}\\
\vspace{3mm}
{\Large\bf CS240 - Fall 2016}\\
\vspace{2mm}
{\Large\bf Assignment 1 Part 2}\\
\vspace{3mm}
\textbf{Due Date: Wednesday September 28 at 5:00pm}
\end{center}

\definecolor{care}{rgb}{0,0,0}
\def\question#1{\item[\bf #1.]}
\def\part#1{\item[\bf #1)]}
\newcommand{\pc}[1]{\mbox{\textbf{#1}}} % pseudocode

Please read
\url{http://www.student.cs.uwaterloo.ca/~cs240/f16/guidelines.pdf}
for guidelines on submission.  Problems 5 -- 8(a) are written
problems; submit your solutions electronically as a PDF with file
name {\tt a01p2wp.pdf} using MarkUs. We will also accept individual
question files named {\tt a01q5w.pdf}, {\tt a01q6w.pdf}, ... , {\tt
a01q8w.pdf} if you wish to submit questions as you complete them.
Submit your solution to 8(b) electronically as a file named {\tt
report.cpp}.

Note: Assignment 1 has been split into 2 parts both worth 2.5\% each. 
Keep in mind that part 1 is due before part 2. 
There are 39 marks for part 1 and 34 marks for part 2.
%%%%%%%%%%%%%%%%%%%%%%%%%%%%%%%%%%%%%%%%%%%%%%%%%%%%%%%%%%%%%
\setcounter{subsection}{4}
\subsection{[2+2+4+4=12 marks]}

Consider the following procedure.

\begin{verbatim}
pre: n is a positive integer
pre: v[1..n] is a binary vector of length n, 
     i.e., each entry is either 0 or 1
foo(v,n)
1.   i := 1;
2.   while i<=n and v[i]=0 do 
3        i := i+1
4    od;
5.   for j from 1 to i do
6.       print("Hello world!")
7.   od;
\end{verbatim}

\begin{itemize}
\part{a} How many inputs are there are of size $n$?
\\\textbf{Solution: } There are totally $2^n$ inputs of size n.
\part{b} What is the worst case number of calls to print?  Give an
exact formula in terms of $n$ and justify your answer by giving an
example of a worst case input of size $n$.
\\\textbf{Solution: }The worst case happens when all entries in v are 0, which takes n+1 iterations. 
\part{c} For $i \in \{1,2,\ldots,n\}$, let $S_i$ denote the subset
of inputs of size $n$ for which the number of calls to print is
$i$.  Describe  and enumerate  $S_i$.
\\\textbf{Solution: }
Before enumerating $S_i$, I will give an simple example. Eg: v = [0, 0, 0, 0, 1, x, x, x, x], where v has size 9 and the first loop will terminate when i goes to the fifth entry in the array, which is 1. Thus, $i = 5$ after the first loop. Therefore, the print is called 5 times totally. Because the first five entries is unchanged, the last four entries can be enumerated. Thus, the enumeration of $S_i$ is $2^{9-5} = 2^4 = 16$.
\\Conclusively, the enumeration of \textbf{$S_i = 2^{n-i}$}.
\part{d} What is the average case number of calls to print?  Derive
an exact closed form formula in terms of $n$.
\\\textbf{Solution: }
\\
\\$T_{A}^{avg}(n) = \frac{1}{|{I:size(I) = n}|}\sum_{I:size(I) = n} T_A(I)$
\\
\\$=\frac{1}{2^n} \sum_{i = 1}^{n} i(2^{n-i})$
\\
\\$=\sum_{i=1}^{n} \frac{i}{2^i}$
\\
\\$=2-(n+2)/2^n$ from A1P4
\end{itemize}
%%%%%%%%%%%%%%%%%%%%%%%%%%%%%%%%%%%%%%%%%%%%%%%%%%%%%%%%%%%%%
\subsection{[5 marks]}
Prove that the following code fragment will always terminate.
\begin{verbatim}
s := 3*n  // n is an integer
while (s>0)
   if (s is even)
      s := floor(s/4)
   else
      s := 2*s
\end{verbatim}
\textbf{Solution:}
\\If s is even, then it takes one operation to get one-fourth of its value. Because there is a $s>0$ condition, if n is always a multiple of four, program will terminate in $\frac{1}{2} \log {n} + 4$ operations, where 4 means when n becomes 1, there are still four operations left, which is $s = 3 * 2 = 6 \rightarrow s = \frac{6}{4} = 1 \rightarrow s = 1*2 = 2 \rightarrow s = \frac{2}{4} = 0$, . Otherwise, if s is always odd, which is the worst case, there is one more operation in each iteration. Thus, it will take $floor(\log{3n})$ operations to complete the program. So, the program will terminate in both the best case and the worst case.
%%%%%%%%%%%%%%%%%%%%%%%%%%%%%%%%%%%%%%%%%%%%%%%%%%%%%%%%%%%%%
\subsection{[5 marks]}
Analyze the following piece of pseudo-code and give a tight bound
(i.e. $\Theta$ notation) on the running time as a function of $n$.it 
Show your work. A formal proof is not required, but you should
justify your answer.
  \begin{algorithme}
        \lign $mystery \leftarrow 0$\\ 
        \lign \pc{for} $i \leftarrow 1$ \pc{to} $3n$ \pc{do}\\
	\lign \>$mystery \leftarrow mystery\times 4$\\
        \lign \> \pc{for} $j \leftarrow 1388$ \pc{to} $2010$ \pc{do}\\
        \lign \>\> \pc{for} $k \leftarrow 4i$ \pc{to} $6i$ \pc{do}\\
        \lign \> \>\> $mystery \leftarrow mystery+k$
  \end{algorithme}
\textbf{Solution:}
\\$\Theta{(\sum_{i=1}^{3n} \sum_{j=1388}^{2010} 2i)} $
\\
\\$= \Theta(\sum_{i=1}^{3n} (2010-1388+1)*2i)$
\\
\\$= \Theta(\sum_{i=1}^{3n} 1246i)$
\\
\\$=\Theta(1246* \frac{(3n+1)*3n}{2})$
\\
\\$=\Theta(n^2)$
%%%%%%%%%%%%%%%%%%%%%%%%%%%%%%%%%%%%%%%%%%%%%%%%%%%%%%%%%%%%%
\subsection{[6+6=12 marks]}
You are given an array $A[0\ldots n-1]$ of integers (not necessarily
distinct) that forms a max-heap of size $n$.
\begin{itemize}
\part{a} Describe an algorithm that takes as input an integer $c$,
not necessarily in the heap, and reports all integers in the heap
that are greater than or equal to $c$.  The running time of your
algorithm should be $O(k)$, where $k$ is the number of integers
reported. Provide a brief explanation for why the running time of
your algorithm is $O(k)$.
\\\textbf{Solution: }
\\I will use the property of max-heap, and use recursion to find all nodes greater than integer $c$. If there is a function $findgreater(maxheap, c)$, the root needs to be checked first. If $root < c$, then return. Otherwise, prints the root, and do findgreater(maxheap$\rightarrow$left, c) and findgreater(maxheap$\rightarrow$right, c) at the same time. When either parts reach a node smaller than $c$, then recursion stops. Otherwise, prints all nodes greater than or equal to $c$.
\part{b} Implement your algorithm from part (a).
Your program should read from {\tt cin} the size $n$, then the $n$
integers in the heap $A[0\ldots n-1]$, and finally the integer $c$,
and then write to {\tt cout} the integers in the heap that are greater
than or equal to $c$. You may assume that every integer in the input
is at least 0 and at most $2^{31}-1$ (so every integer will fit into a
variable of type \verb|int|).

Every integer in the input and output should be on a separate line.
So for instance if the input consists of the following lines:

\begin{center}\fbox{\begin{minipage}{.5\textwidth}
  \tt
  5  \\
  17 \\
  15 \\
  13 \\
  10 \\
  3 \\
  12 
\end{minipage}}\end{center}

then your program should print out the integers \verb|17|, \verb|15|,
and \verb|13| in any order (and on separate lines).

Submit the code for your \verb|main| function, along with any helper
functions, in a file called \verb|report.cpp|.

\end{itemize}
\end{document}
